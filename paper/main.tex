\documentclass{article}
\usepackage{graphicx} 
\usepackage{amsmath}
\usepackage{amssymb} 
\usepackage{hyperref}
\usepackage{natbib}
\usepackage{parskip}
\bibliographystyle{apalike}
\usepackage[utf8]{inputenc}
\usepackage[T1]{fontenc}
\usepackage[
  a4paper,
  left=25mm,
  right=25mm,
  top=25mm,
  bottom=25mm
]{geometry}

\title{Misinformation Spread on Social Media: \\ A Game-Theoretic Approach}   
\author{Felipe Manzi Cherryl Chico Olsa Berani}
\date{November 2025}


\begin{document}

\maketitle
\begin{abstract}
The spread of misinformation on social networks threatens societal stability by shaping public opinion and collective action. This paper develops a game-theoretic model in which agents, motivated by reputational rewards, interact with information and update their actions over time. These actions are applied to a structured social network where diffusion is governed by agents’ influence and reach. By capturing how individual decisions aggregate across diverse topologies, the model quantifies contamination rates within neighborhoods after repeated updates, highlighting how utility-driven behaviors shape the velocity and extent of misinformation spread.

\end{abstract}

\section{Introduction}

The spread of misinformation on social media poses substantial challenges to public discourse and collective decision-making. While exposure and diffusion dynamics have been extensively studied—particularly on platforms such as Facebook, where recent evidence suggests a decline in the circulation of false content \citep{allcott2019trends}, much less is known about alternative communication environments like Telegram. Unlike Facebook, Twitter/X, and other algorithm-driven platforms, Telegram does not rely on a personalized recommendation system. Information spreads primarily through channels, groups, and direct forwarding, rather than through algorithmic curation. This absence of recommendation algorithms provides a cleaner environment to observe how misinformation propagates, as diffusion is less confounded by opaque platform-level ranking mechanisms.

In this study, we aim to investigate how this distinct institutional setting shapes the propagation of misinformation. A central question guides our analysis: what motivates individuals to share false information, and how do their social relationships and reputational concerns influence this behavior?

Our work builds upon the economic foundations of information behavior. First, reputational concerns have long been recognized as fundamental drivers of human behavior: individuals care about how their actions affect how they are perceived by others \citep{benabou2006incentives, benabou2011identity}. This logic directly parallels models of media behavior, where information senders strategically choose what to transmit because audiences update beliefs about the sender's credibility. In particular, the reputation-driven framework of \citet{gentzkow2006media}—in which media outlets condition reporting on how it affects perceived accuracy—serves as a microfoundational motivation for how we model reputation among individuals in messaging platforms.

Second, empirical research finds that news sharing is shaped not only by beliefs but also by social approval, status motives, and identity-driven incentives \citep{talwar2020fake, wu2025motivations, vellani2024motives}. Evidence from WhatsApp further shows that reputational mechanisms matter: forwarded-message labels influence perceptions of credibility and willingness to share \citep{tandoc2022forwarded}, and exposure to fact-checking reduces subsequent forwarding \citep{reis2020whatsapp}. Work on Telegram additionally documents high-concentration misinformation ecosystems in channels shaped by tight-knit community norms \citep{herasimenka2022telegram}.

Finally, our approach is closely related to recent advances in misinformation modeling. \citet{acemoglu2024misinformation} develop an equilibrium framework in which agents weigh social utility against the risk of reputational loss (“being called out”), generating endogenous filter bubbles. Dynamic network approaches such as \citet{yilmaz2022deep} highlight how strategic interactions and influence-maximizing dynamics shape diffusion on evolving topologies.

While these studies provide deep insight into misinformation and strategic behavior, most focus on equilibrium characterizations, platform-level incentives, or adversarial spread. Our research departs from this by explicitly modeling multi-round diffusion where an agent’s reputation and subjective beliefs evolve endogenously and shape future propagation. We focus on three primary questions: 


\begin{enumerate}
    \item How does an agent's subjective belief regarding the veracity of a message, coupled with the value they place on their personal reputation, affect the initial propensity and subsequent widespread dissemination of misinformation?
    \item In a dynamic setting, how is misinformation contained within local neighborhoods, or conversely, how is it allowed to spread more widely across the network upon multiple rounds of agents updating their information and actions?
    \item How does the velocity and final extent of misinformation spread vary across different, structured topologies of social networks, such as scale-free, random, or small-world networks?
\end{enumerate}

To answer these research questions, our strategy employs two  components: (i) a formalization of the agents' forwarding and updating decisions based on game theory, and (ii) large-scale agent-based simulations designed to trace message diffusion across social networks. Specifically, we focus on platforms such as WhatsApp and Telegram, where forwarding behavior plays a central role in information propagation. We will conduct these simulations across a spectrum of network topologies and agent types to produce quantitative results on the rate of spread and provide a comprehensive description of how information contamination evolves in various structural and behavioral scenarios.

\section{Model and Environment}

We study information diffusion in a social network environment similar to WhatsApp and Telegram, referred to here as WT platforms.
These platforms provide a natural setting for our analysis, as they are structured around group 
interactions where users exchange, evaluate, and forward messages. Within this environment, the 
core components of our model involve: (i) the messages circulating through groups, (ii) the agents 
(users) who receive and act on these messages, and (iii) the relationships among agents that define 
group membership and interaction patterns. We formally define each of these components below.

\subsection{Messages}
In WT platforms, a single message $m$ is the basic unit of information that users receive, 
evaluate, and decide whether to forward. Each message carries an ideological slant, reflecting 
the political or social bias embedded in its content, and a truth value, indicating whether the 
information is factually correct. Formally, each message is represented by its ideological slant and truthfulness:
\begin{align}
\m &\sim (\bias, \truth) \qquad  
\end{align}
where
\begin{align}
\bias &= 2x - 1,\quad x \sim \text{Beta}(\alpha, \beta), \bias \in [0,1] \\
\truth &\sim \text{Bernoulli}(q), \quad \truth \in {0,1}
\end{align}

We simplify ideology into two opposing poles. The parameter $\bias$ captures the strength of 
the message’s leaning, with values closer to $-1$ or $1$ indicating stronger alignment with 
one of the extremes.  A value of $\beta > \alpha$ skews the distribution towards  $1$ , while $\beta < \alpha$ skews it towards $-1$. We assume that a message is only either true, $\truth=1$ and false $\truth=0$. The parameter $q$ captures the probability of getting a true message. 
\subsection{Agents}

The set of agents $N$ represents users in the WT network. Each agent $\agent \in N$ is a strategic 
participant in group interactions, also characterized by an ideological bias $\bias$ representing 
the agent’s own slant, and a reputation value $\R \in \mathbb{R}$, reflecting their credibility 
based on past forwarding behavior. Thus,
\begin{equation}
\agent \sim (\bias, \R)
\end{equation}

where $\bias$ is similarly modeled as in Equation 2 and 
\begin{equation}
 \R \sim \mathcal{N}(\mu, \sigma^2), \quad \R \in \mathbb{R}
\end{equation}

The reputation $\R$ is initialized from a normal distribution where a higher $\mu$ represents higher average reputation and the variance $\sigma^2$ captures the diversity of reputations across identically distibuted users.

Within WT groups, agents evaluate the truthfulness of incoming messages and select an action according to expected utility. Depending on the actual truthfulness of the message and the 
agent’s choice, their reputation updates over time. We describe each of these components below.

\subsubsection{Belief Formation}
In WT platforms, the truthfulness of a message is often not directly observable. Agents therefore 
form beliefs using two cues: the ideological slant of the message relative to their own bias, and 
the sender’s reputation. This structure aligns with evidence that trust and credibility in group 
messages depend on ideological proximity and the perceived reliability of the source 
\citep{talwar2020fake, vellani2024motives}. The belief of agent $i$ about the truthfulness of 
message $m$ sent by agent $j$ is:
\begin{equation}
\belief =
    \begin{cases}
    \truth, & \text{if truth is observable} \\
    \widehat{\truth}, & \text{if truth is not observable}
    \end{cases}
\end{equation}
where
\begin{align}
\widehat{\truth} &=
    \begin{cases}
        1, & \text{if } \phi \cdot \sigma \geq 0.5 \\
        0, & \text{if } \phi \cdot \sigma < 0.5
    \end{cases}, \\
\phi(\bias_i, \bias_m) &= 1 - |\bias_i - \bias_m|, \quad \phi \in [0,1] \\
\sigma(\R) &= \frac{1}{1 + e^{-\R}}, \quad \sigma \in \{0,1\}
\end{align}

The function $\widehat{\truth}$ estimates the agent's belief on the truthfulness of the message according to \textit{ideological proximity} represented by the function $\phi$. The ideological proximity is scaled by the reputation of the sender by a monotone transformation of $\R$ function $\sigma$. If $\widehat{\truth_m} > 0.5$ then the $belief = 1$ and 0 otherwise.
Intuitively, in WT groups, agents are more likely to believe messages when they originate from ideologically aligned senders with high reputations.

\subsubsection{Action Choice}
After forming $\belief$, agent $\agent$ chooses whether to forward the message to their WT contacts. The action set is
\[
\action \in \{0,1\},
\]
where $\action = 1$ denotes forwarding and $\action = 0$ denotes ignoring.  

Forwarding decisions in WT platforms reflect three incentives: expressive satisfaction from 
sharing ideologically aligned content, reputational gains if the message is true, and reputational 
losses if it is false. Forwarding also carries a fixed cost, such as the effort or risk of 
spreading misinformation. Ignoring yields no utility. Thus, the expected utility is:
\begin{equation}
\mathbb{E}[u] =
\begin{cases}
\omega \phi
+ \gamma\belief
- \delta (1 - \belief)
- \cost, & \text{if } \action = 1 \\
0, & \text{if } \action = 0.
\end{cases}
\end{equation}

where $\omega, \gamma, \delta \in [0,1]$ and $\cost > 0$.
Here, $\omega$ represents the benefit from forwarding ideologically aligned messages, 
$\gamma$ the reputational gain from forwarding a true message, 
$\delta$ the reputational penalty from forwarding a false one, 
and $k$ the fixed cost of forwarding. Because agents in WT groups often receive the same 
message from multiple senders, the overall utility is the average across all sources:
\[
\mathbb{E}[u_{i}] = \frac{1}{\|S_i\|} \sum_{j \in S_i} \mathbb{E}[u_{ij}],
\]
where $S_i$ is the set of senders from whom agent $i$ received the message at time $t$. 
The agent forwards if $\mathbb{E}[u_i] \geq 0$.

\subsubsection{Reputation Update}
We assume reputation is the guiding principle for agents in the WT network. 
Empirical studies of WhatsApp and Telegram show that users rely on sender credibility, 
and reputational cues strongly shape forwarding behavior 
\citep{pasquetto2022social, herasimenka2022telegram}. 
Accordingly, our model treats reputation as the central mechanism of credibility: it rises 
when agents forward true messages and falls when they forward false ones. Formally,

\begin{equation}
R_i^{t+1} = R_i^t 
+ \gamma \cdot \mathbf{1}\{\action_i^t = 1 \cap \truth_t = 1\}
- \delta \cdot \mathbf{1}\{\action_i^t = 1 \cap \truth_t = 0\}
\end{equation}

 Reputation evolves only through the correctness of forwarding actions. Ignoring a message leaves reputation unchanged.

\subsection{Social Network}
Social networks on WT platforms are organized around 
groups of users who interact repeatedly. In our representation, each agent corresponds 
to a node, and communication ties (e.g., channel membership) are modeled as edges. Note that the edge is directed and one way to model from one agent to another mimicking the action of forwarding a message. Empirical studies show that these platforms are characterized by dense clusters 
of interpersonal ties within groups, alongside weaker spillovers across groups 
\citep{herasimenka2022telegram}. 

We model the social network as a directed graph $G = (N, E)$, where $N$ is the set of agents (nodes) and $E$ is the set of directed edges representing communication ties (e.g., channel memberships). The network is generated using a stochastic block model to capture the clustered group structure typical of WT platforms. Formally, we partition agents into $K$ latent groups $\{G_1, G_2, \ldots, G_K\}$. The probability of a directed edge from agent  $i$ to agent $j$ depends on their group memberships:

\begin{equation}
\Pr((i,j)\in G) =
\begin{cases}
p_{\text{in}}, & \text{if } i,j \text{ share at least one group}, \\
p_{\text{out}}, & \text{otherwise},
\end{cases}
\quad \text{with } p_{\text{in}} > p_{\text{out}} > 0.
\end{equation}

Here, $p_{in}$ is the probability of a directed edge between agents within the same group, while $p_{out}$ is the probability of an edge between agents in different groups. This structure captures the tendency for users on WT platforms to form tightly-knit groups with frequent interactions, while still allowing for cross-group connections that facilitate broader information diffusion.


\section{Data Generating Process and Simulation}
\input{simulation.tex}

\section{Outcome Measurement}

To evaluate how misinformation propagates across different network structures and agent types, we record a series of outcome metrics at each round $t$ of the simulation. These measures are designed to capture both the \emph{extent} and the \emph{dynamics} of diffusion, including whether false messages remain contained within local communities or spread widely across the network.

\subsection{Reach}

We define the \textit{reach} of message $m_t$ as the fraction of agents who received the message through any chain of forwards:
\[
R_t = \frac{|\{ i : \text{agent } i \text{ received } m_t \}|}{N}.
\]
This metric captures how far a message travels across the network.

\subsection{Forwarding Rate}

Among those who received the message, we compute the \textit{forwarding rate}:
\[
F_t = \frac{\sum_{i: i \text{ received } m_t} a_{it}}{|\{ i : \text{received } m_t \}|},
\]
where $a_{it} \in \{0,1\}$ denotes agent $i$'s forwarding decision.  
This measure reflects the behavioral response to the message in round $t$.

\subsection{Misinformation Contamination}

To track the spread of false messages, we compute the \textit{false-forward rate} when $\theta_t = 0$:
\[
FF_t = \frac{\sum_{i: i \text{ received } m_t} a_{it} \cdot \mathbf{1}\{\theta_t = 0\}}
{|\{ i : \text{received } m_t \}|}.
\]
We also record cumulative contamination over time as the total number of agents ever exposed to false messages.

\subsection{Velocity of Spread}

The \textit{velocity} of diffusion is defined as the time required for a message to reach a given share of the population.  
For a threshold $\tau \in (0,1)$, define:
\[
V_{\tau}(t) = \min \{ s \geq t : R_s \geq \tau \}.
\]
A lower $V_{\tau}$ indicates faster spread.

\subsection{Community Containment versus Spillover}

Each agent belongs to one or more communities in the overlapping group structure.  
For each receiving agent $i$, let $\mathcal{C}(i)$ denote the set of communities to which $i$ belongs. We measure the degree of within-community concentration of diffusion through:
\[
C_t = \frac{1}{|\{ i : \text{received } m_t \}|}
\sum_{i: \text{received } m_t}
\mathbf{1}\{ \exists\, j \text{ sender with } 
\mathcal{C}(i) \cap \mathcal{C}(j) \neq \emptyset \}.
\]
A high $C_t$ indicates that diffusion remains mostly inside local group ``bubbles,'' whereas a low $C_t$ indicates cross-community spillovers.

\subsection{Reputation Dynamics}

For each agent type $\tau$, we track the mean reputation over time:
\[
\bar{R}_{\tau}(t) =
\frac{1}{|\{i : \tau(i) = \tau\}|}
\sum_{i: \tau(i)=\tau} R_i^t.
\]
This allows us to study how exposure to true and false messages shapes the reputational distribution of influencers, regular users, and automated accounts.

\subsection{Aggregation Across Rounds and Network Draws}

For each metric $\{R_t, F_t, FF_t, C_t\}$, we compute:

\begin{itemize}
    \item \textbf{Round-level averages:} $\frac{1}{T}\sum_{t=1}^T X_t$.
    \item \textbf{Cumulative measures:} e.g., cumulative false exposure.
    \item \textbf{Comparative statics:} comparison across network structures (scale-free, small-world, overlapping SBM) and agent-type parameterizations.
\end{itemize}

These aggregated measures provide a comprehensive characterization of misinformation diffusion under different structural and behavioral assumptions.


\bibliography{references}





\end{document}
